\documentclass{article}
\usepackage{lawbrief}
\usepackage{makeidx}
\makeindex

\begin{document}


\NewBrief{subject={Separate but Equal, Education},
        plaintiff={Brown},
        defendant={Board of Education},
        citation={347 U.S. 483 (1954)},
        facts={Segregation in public education, under separate-but-equal laws.\\
               Brown challenged the constitutionality of segregation in public schools.},
        procedure={District court upheld segregation under \hyperref[case:plessy]{\textit{Plessy v. Ferguson}}},
        issue={Does segregation violate the Equal Protection Clause of the 14th Amendment?},
        holding={Yes. Separate educational facilities are inherently unequal},
        reasoning={Segregation generates a feeling of inferiority, harmful to education},
        opinions={Concurring opinions emphasized psychological harm},
        label={case:brown}
}

See also \hyperref[case:brown]{\textit{Brown v. Board of Education}}.

\NewBrief{subject={Separate but Equal, Train},
        plaintiff={Plessy},
        defendant={Ferguson},
        citation={163 U.S. 537 (1896)},
        facts={Plessy, a man of mixed race, sat in a "whites only" train car and was arrested under Louisiana's segregation laws.},
        procedure={State courts upheld the conviction; Plessy appealed to the U.S. Supreme Court.},
        issue={Does state-imposed racial segregation violate the Equal Protection Clause of the 14th Amendment?},
        holding={No. Laws requiring racial segregation do not imply the inferiority of either race and are therefore constitutional.},
        reasoning={The Court held that “separate but equal” facilities did not violate the Constitution, legitimizing segregation.},
        opinions={Justice Harlan dissented, warning the decision would become as infamous as Dred Scott.},
        label={case:plessy}
}

See also \hyperref[case:plessy]{\textit{Plessy v. Ferguson}}.


\printindex
\end{document}
