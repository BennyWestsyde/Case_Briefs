
            \documentclass[../tex_src/CaseBriefs.tex]{subfiles}
            \usepackage{lawbrief}
            \begin{document}
            \NewBrief{subject={False Imprisonment},
                    plaintiff={Enright},
                    defendant={Groves},
                    citation={39 Colo.App.\ 39, 560 P.2d 851 (1977)},
                    course={Torts},
                    facts={The Plaintiff (Enright) lived in a town with a "dog leash" ordinance where Defendant (Groves) was a police officer.\ Plaintiff's dog was found to be off its leash and defendant found out.\ Defendant demanded plaintiff's drivers license, to which plaintiff refused.\ Defendant demanded it once more, informing Plaintiff that if it wasn't produced she would be going to jail.\ After Plaintiff refused again, Defendant arrested her.},
                    procedure={Trial Court: In favor of Plaintiff (\$500 Actual, \$1000 Exemplary)\\
Appeal: In favor of Plaintiff},
                    issue={Is it false imprisonment if the police arrest someone for a violation which the officer does not have probable cause to pursue?},
                    holding={No; Ruling Affirmed},
                    principle={The defense based on arrest for and conviction of a specific offense is invalid if the arresting "offense" is beyond of the scope of the original offense?},
                    reasoning={},
                    opinions={},
                    label={case:EnrightVGroves},
                    notes={The elements of false arrest are...\\
- a willful detention of a person\\
- against the consent of the person\\
- without the authority of the law}
            }
            \end{document}
        