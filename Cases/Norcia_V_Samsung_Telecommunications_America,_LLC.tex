
            \documentclass[../CaseBriefs.tex]{subfiles}
            \usepackage{lawbrief}
            \begin{document}
            \NewBrief{subject={Implied Contract},
                    plaintiff={Norcia},
                    defendant={Samsung Telecommunications America, LLC},
                    citation={845 F.3d 1279 (2017)},
                    course={},
                    facts={Plaintiff purchased a phone manufactured by Defendant. Along with the phone was a brochure consisting of a "Standard Limited Warranty" and a EULA, within which was an arbitration clause. The clause stated that the purchaser had the right to opt out of the arbitration clause by providing notice to Defendant within 30 days of the purchase. However, when plaintiff purchased the phone, they carried the phone out and left the brochure and box.

Plaintiff later filed a class action complaint against Defendant, claiming that the specs of the phone had been misrepresented in its advertising. Defendant replied by moving to compel arbitration. },
                    procedure={Trial Court: In favor of Plaintiff
Appeal: In favor of Plaintiff (Motion Denied)},
                    issue={Does defendant's inaction constitute agreement to the arbitration clause?},
                    holding={No; Motion Denied},
                    principle={An offerer must provide some means of manifesting an offeree's intent to use silence or inaction as means of accepting an agreement or be able to prove such silence or inaction bestowed a benefit on the offeree},
                    reasoning={Defendant made no indication that there was any expectation for plaintiff to agree to the terms as laid out in the brochure. Further, no benefits would be gained or lost by opting out of arbitration. Therefore, inaction on the part of the plaintiff cannot be judged as agreement to the terms because },
                    opinions={},
                    label={case:NorciaVSamsung},
                    notes={}
            }
            \end{document}
        