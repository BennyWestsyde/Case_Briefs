
            \documentclass[../tex_src/CaseBriefs.tex]{subfiles}
            \usepackage{lawbrief}
            \begin{document}
            \NewBrief{subject={Intentional Infliction of Emotional Distress},
                    plaintiff={Slocum},
                    defendant={Food Fair Stores of Florida},
                    citation={100 So.2d 396 (1958)},
                    course={Torts},
                    facts={While Plaintiff was shopping at Defendant's store, Plaintiff alleges that she was insulted by one of defendant's employees.\ This insult "},
                    procedure={Trial Court: In favor of Defendant (Motion to Dismiss based on lack of Cause of Action Granted)\\
Appeal: In favor of Defendant (Motion to Dismiss Affirmed)},
                    issue={Did the insulting language that plaintiff claims caused her injury constitute an actionable invasion of a legally protected right?},
                    holding={No; Motion to Dismiss Affirmed},
                    principle={A distinction must be drawn between "conduct likely to cause mere 'emotional distress' and that causing 'severe emotional distress'"},
                    reasoning={},
                    opinions={},
                    label={case:SlocumVFoodFair},
                    notes={It is fo rthe court to determine, in the first instance, whether the conduct would reasonably be regarded as outrageous.\ Where reasonable men may differ, it is left to the jury as a question of fact.}
            }
            \end{document}
        