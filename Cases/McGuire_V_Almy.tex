
            \documentclass[../CaseBriefs.tex]{subfiles}
            \usepackage{lawbrief}
            \begin{document}
            \NewBrief{subject={Battery, Intent},
                    plaintiff={McGuire},
                    defendant={Almy},
                    citation={297 Mass. 323, 8 N.E.2d 760. (1937)},
                    facts={The plantiff was employed to take care of the defendant as a registered nurse. The plantiff was on "24 hour duty" watching the defendant. One day, the plantiff heard commotion coming from the defendant's room and found that she had broken her furniture and was threatening anyone who would come into the room with the leg of a table. When the plantiff entered the room to disarm the defendant and remove the debris, the defendant struck her in the head.},
                    procedure={},
                    issue={Can a person who is considered clinically insane be held liable for torts?},
                    holding={Yes; Judgement for the plantiff},
                    principle={In so far as a particular intent would be liable to hold a normal person liable, an insane person, in order to be liable, must have been capable of entertaining the same intent as well as, in fact, entertaining it.},
                    reasoning={The intent of the defendant is crucial in determining liability of a tort, and therefore for a mentally ill defendant the question should be whether they have the capability to have intent as well as whether they actually have the intent.},
                    opinions={},
                    label={case:McGuireVAlmy},
                    notes={At the time, it was a case without precident as to whether a mentally ill person could be held liable for torts. }
            }
            \end{document}
        