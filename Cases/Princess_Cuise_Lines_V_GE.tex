
            \documentclass[../tex_src/CaseBriefs.tex]{subfiles}
            \usepackage{lawbrief}
            \begin{document}
            \NewBrief{subject={Warranty, Agreement},
                    plaintiff={Princess Cuise Lines},
                    defendant={GE},
                    citation={22 Ill.143 F.3d 828 (1998)},
                    course={Contracts},
                    facts={Plaintiff hired Defendant to do service repairs on their boat.\ Plaintiff sent a contract to Defendant that quoted the price for \$260,000 with a warranty.\ Defendant responded by offering to do the work for \$231,925 with no warranty.\ Several days later, defendant sent an acceptance of the offer they had previously and there was no reply.\ However, Plaintiff did bring the boat to the service area and service commenced.\ Due to faulty repair to the rotor, Plaintiff had to cancel a cruise and Plaintiff sued.\ },
                    procedure={Trial Court: In favor of Plaintiff\\
Appeal: In favor of Defendant},
                    issue={Does the UCC apply to the current contract?},
                    holding={No; Reversed and Remanded},
                    principle={A contract that primarily revolves around services must utilize common law rather than the UCC},
                    reasoning={},
                    opinions={},
                    label={case:PrincessVGe},
                    notes={Predominant Purpose Test:\\
Was the predominant purpose of the contract to do a service or order a good? (i.e.\ if the contract was to do services and buying goods is just part of that then the services are predominant.\ Otherwise, if the purchase of the goods was the primary purpose and services where incidental then the purchase would be primary)}
            }
            \end{document}
        