
            \documentclass[../CaseBriefs.tex]{subfiles}
            \usepackage{lawbrief}
            \begin{document}
            \NewBrief{subject={Negligence, Car, Reasonable Care},
                    plaintiff={Cohen},
                    defendant={Petty},
                    citation={62 App.D.C. 187, 65 F.2d 820 (1933)},
                    course={Torts},
                    facts={The plaintiff was riding in a car driven by the defendant, when the defendant was suddenly stricken by an illness that resulted in them passing out behind the wheel. \\
The defendant testified that he knew himself to be in good help and had never fainted before. \\
The defendant wasn\'t driving recklessly and he did not feel ill until moments before he passed out.},
                    procedure={Trial Court: In favor of Defendant\\
Appeal: In favor of Defendant},
                    issue={Can a defendant struck with a sudden and unexpected illness that results in damages to a plaintiff be held liable?},
                    holding={No; Previous Ruling Affirmed},
                    principle={Unexpected and accidental \"acts of god\" cannot be used to hold someone liable for negligence},
                    reasoning={There was no evidence presented that would indicate that any reasonable level of care could have prevented the events from happening, therefore it is unreasonable to hold the defendant liable for negligence},
                    opinions={},
                    label={case:CohenVPetty},
                    notes={}
            }
            \end{document}
        