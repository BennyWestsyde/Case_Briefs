
            \documentclass[../CaseBriefs.tex]{subfiles}
            \usepackage{lawbrief}
            \begin{document}
            \NewBrief{subject={False Imprisonment},
                    plaintiff={Parvi},
                    defendant={City of Kingston},
                    citation={41 N.Y.2d 553, 362 N.E.2d 960, 394 N.Y.S.2d 161 (1977)},
                    course={Torts},
                    facts={The defendant (The City of Kingston Police) responded to a complaint in which two brothers were fighting and plaintiff (Parvi) was trying to calm them down. According to the defendants, all three were showing \"the effects of alcohol.\" When plaintiff told the police he didn\'t have anywhere to go, they took him outside the city limits and left him there to \"dry out.\" Plaintiff then wandered into traffic and was hit by a car.},
                    procedure={Trial Court: In favor of defendant\\
Appeal: In favor of defendant\\
Supreme Court: In favor of plaintiff},
                    issue={If the plaintiff was too drunk to have a recollection of their confinement, does that undermine their cognisense of their captivity? },
                    holding={No; Reversed and remanded},
                    principle={there is no liability for intentionally confining another unless the person physically restrained knows of the confinement or is harmed by it},
                    reasoning={},
                    opinions={},
                    label={case:ParviVKingston},
                    notes={the plaintiff must show that: \\
(1) the defendant intended to confine him, \\
(2) the plaintiff was conscious of the confinement, \\
(3) the plaintiff did not consent to the confinement and \\
(4) the confinement was not otherwise privileged}
            }
            \end{document}
        