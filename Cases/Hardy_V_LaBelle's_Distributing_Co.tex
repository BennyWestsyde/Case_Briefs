
            \documentclass[../CaseBriefs.tex]{subfiles}
            \usepackage{lawbrief}
            \begin{document}
            \NewBrief{subject={False Imprisonment},
                    plaintiff={Hardy},
                    defendant={LaBelle\'s Distributing Co},
                    citation={203 Mont.\ 263, 661, P.2d 35 (1983)},
                    course={Torts},
                    facts={Plaintiff (Hardy) was hired by Defendant (LaBelle\'s Distributing Co) as a sales clerk in the jewelry department.\ On December 9, another employee thought they saw plaintiff steal a watch from a display and reported that suspicion to the assistant manager.\ The next day, the assistant manager told plaintiff that he wanted to give her a tour of the office, but then led her to a room with police officers and the loss prevention manager.\ They told her about the accusation, which she denied, and asked her to take a poligraph, which she accepted and passed.\ After the interaction, plaintiff filed a suit for false imprisonment based on her being brought to the room and expected to stay under false pretenses.},
                    procedure={Trial Court: In favor of Defendant\\
Appeal: In favor of Defendant},
                    issue={Does a situation in which someone feels compelled to stay somewhere constitute false imprisonment?},
                    holding={No; Ruling Affirmed},
                    principle={The two key elements of false imprisonment are the restraint of an individual against their will, by either acts or merely by words which they fear to disregard, and the unlawfullness of such restraint},
                    reasoning={Given that plaintiff admitted that she would have wanted to go clear up the accusations if she had known why she was being brought to the office, she cannot be said to have been restrained.\ Particularly because she was never told not to leave directly or otherwise.},
                    opinions={},
                    label={case:HardyVLaBelle},
                    notes={}
            }
            \end{document}
        