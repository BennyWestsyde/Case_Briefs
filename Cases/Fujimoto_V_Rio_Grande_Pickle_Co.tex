
            \documentclass[../tex_src/CaseBriefs.tex]{subfiles}
            \usepackage{lawbrief}
            \begin{document}
            \NewBrief{subject={Breach of Contract},
                    plaintiff={Fujimoto},
                    defendant={Rio Grande Pickle Co},
                    citation={414 F.2d 648 (1969)},
                    course={Contracts},
                    facts={Plaintiffs worked at Defendant\'s pickle company.\ Defendant offered a contract to plaintiffs that added an annual bonus of 10\% of the company\'s earnings.\ When plaintiffs asked for the contract in writing, one was prepared and sent to them.\ The contract provided no details on how to accept, so plaintiffs simply signed and kept the document.\ After plaintiffs left the company, they filed for breach of contract claiming that they had never been provided the bonuses.\ Defendant objected on the grounds that the contracts had never been returned to them, and therefore were not accepted.},
                    procedure={Trial Court: In favor of Plaintiff (\$8,964.25 in damages)\\
Appeal: In favor of Plaintiff (Affirmed)},
                    issue={Can an offer, which by its terms does not specify the means by which they could be accepted, be accepted by a mode other than the return of the signed instruments?},
                    holding={Yes; Ruling affirmed},
                    principle={A contract may require the acceptor to relay that acceptance by any means; however, in the absence of such a requirement, an acceptors actions (i.e.\ continuing to work for the company and not mentioning the negotiations again) constitute acceptance.},
                    reasoning={},
                    opinions={},
                    label={case:FujimotoVRioGrandePickle},
                    notes={}
            }
            \end{document}
        