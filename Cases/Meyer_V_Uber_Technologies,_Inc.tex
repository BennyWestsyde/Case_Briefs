
            \documentclass[../CaseBriefs.tex]{subfiles}
            \usepackage{lawbrief}
            \begin{document}
            \NewBrief{subject={Browsewrap, Mutual Assent},
                    plaintiff={Meyer},
                    defendant={Uber Technologies, Inc},
                    citation={868 F.3d 66 (2017)},
                    course={Contracts},
                    facts={The plaintiff (Meyer) signed up for and used the defendant's (Uber's) app to request 10 rides. Afterwards, Meyer accused Uber of engaging in price fixing and prepared to try to take Uber to trial over the charge. Uber requested a motion to compel arbitration per their terms of service agreement. Meyer objected to this on the grounds that the terms of service were not made obvious enough when he was registering for the app. \\
The terms of service where situated as "browsewrap" terms meaning that a user was given the option to view the terms, but where not forced to, and would "sign" the contract based on clicking "Register".\\
The district court sustained this objection and denied the motion to compel arbitration.},
                    procedure={Pre-trial: Uber motion to compel arbitration -> District court denies\\
Appeal: Uber appeals this denial -> Denial overruled (Vacated and remanded)},
                    issue={Does a "browsewrap" terms of service agreement provide users enough opportunity to read and understand the contractual nature of using a website or app?},
                    holding={Yes; Vacated and remanded},
                    principle={A reasonable and competent user should be able to understand the contractual and transactional nature of using websites with a "browsewrap" terms of service},
                    reasoning={Inasmuch as consumers are regularly and frequently confronted with non-negotiable contract terms, particularly when entering into transactions using the internet, the presentation of these terms at a place and time that the consumer will associate the initial purchase or enrollment, or the use of, the goods or services from which the recipient benefitsat least indicates to the consumer that he or she is taking such goods or employing such services subject to additional terms and conditions that may one day affect him or her},
                    opinions={},
                    label={case:MeyerVUber},
                    notes={}
            }
            \end{document}
        