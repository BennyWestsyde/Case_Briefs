
            \documentclass[../tex_src/CaseBriefs.tex]{subfiles}
            \usepackage{lawbrief}
            \begin{document}
            \NewBrief{subject={Negligence, Tresspass (common law)},
                    plaintiff={Brown},
                    defendant={Kendall},
                    citation={60 Mass.\ (6 Cush.) 292 (1850)},
                    course={Torts},
                    facts={The plaintiff and defendant\'s dogs where fighting and, in order to break up the fight, the defendant picked up a stick to hit the dogs to separate them.\ While raising the stick, the defendant struck the plaintiff, who was standing behind him, in the eye.\ \\
\\
The defendant was acting lawfully, and there was no indication of unlawful intent},
                    procedure={Jury Trial: In favor of Plaintiff\\
Appeal: In favor of Defendant (New Trial Ordered)},
                    issue={Does the defendant hold the burden of proof when an unintended consequence results from a lawful act without unlawful intentions?},
                    holding={No; New Trial Ordered},
                    principle={The plaintiff holds the burden of proving that a defendant acted either unlawfully or carelessly},
                    reasoning={If an act is lawful and a purely accidental injury arrises, the actor cannot be held liable},
                    opinions={},
                    label={case:BrownVKendall},
                    notes={}
            }
            \end{document}
        