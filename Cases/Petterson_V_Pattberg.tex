
            \documentclass[../tex_src/CaseBriefs.tex]{subfiles}
            \usepackage{lawbrief}
            \begin{document}
            \NewBrief{subject={Unilateral Contract, Revocation},
                    plaintiff={Petterson},
                    defendant={Pattberg},
                    citation={248 N.Y.\ 86 161 N.E.\ 428 (1928)},
                    course={Contracts},
                    facts={The plaintiff (Petterson) was the owner of 5301 Sixth Avenue in Brooklyn.\ Defendant (Pattberg) was the owner of a bond executed by Plaintiff, secured by the mortgage on Plaintiff's property.\ Plaintiff sought to pay off \$780 of his loan in cash and approached Defendant with the offer.\ Defendant agreed, offering the unilateral contract in a letter.\ However, when Plaintiff approached Defendant with the cash, Defendant rebuffed him, informing him that he had sold Plaintiff's debt and so that deal was off the table.},
                    procedure={Trial Court: In favor of Plaintiff\\
Appeal: In favor of Defendant},
                    issue={Can an offerer revoke an offer to accept payment so long as that payment hasn't yet been tendered?},
                    holding={Yes; Ruling Reversed},
                    principle={Such a contract can be revoked up until the point were the offerer recieves and accepts their payment},
                    reasoning={},
                    opinions={Lehman, J: Defendant's letter constituted a promise to accept payment if Plaintiff comes and offers it.\ The distinction between "offering payment" and "tendering payment" in this situation can only confuse the results.\ If the offer was to accept payment, but allows the offerer to deny that the promise is binding until after they have accepted it then the offer doesn't mean anything.\\
},
                    label={case:PettersonVPattberg},
                    notes={}
            }
            \end{document}
        